\section{Wnioskowanie przybliżone}

Wnioskowanie przybliżone to technika stosowana w systemach ekspertowych umożliwiająca podejmowanie decyzji w warunkach niepełnej, nieprecyzyjnej lub niepewnej informacji. W przeciwieństwie do klasycznego wnioskowania dedukcyjnego, które operuje na faktach jednoznacznych i ściśle zdefiniowanych regułach logicznych, wnioskowanie przybliżone pozwala na elastyczne interpretowanie danych wejściowych, co zbliża sposób działania systemu do ludzkiego rozumowania.

W systemie ekspertowym wspierającym decyzje na rynku nieruchomości (kupno, sprzedaż, wynajem), mechanizmy wnioskowania przybliżonego są szczególnie przydatne w sytuacjach, gdy:
\begin{itemize}
    \item użytkownik nie potrafi precyzyjnie określić swoich preferencji (np. co oznacza „dobra lokalizacja” lub „przystępna cena”),
    \item dane rynkowe są częściowo niekompletne, nieaktualne lub niejednoznaczne,
    \item wymagane jest uwzględnienie subiektywnych ocen i preferencji w procesie rekomendacyjnym.
\end{itemize}

\subsection*{Przykłady zastosowania wnioskowania przybliżonego w regułach}

W systemie zastosowano reguły o charakterze jakościowym i nieprecyzyjnym, w tym:
\begin{itemize}
    \item reguły oparte na wartościach progowych (np. „mieszkanie jest duże, jeśli ma więcej niż 80 m²”),
    \item klasyfikacje lingwistyczne (np. „dobra lokalizacja”, „niska cena”, „nowe mieszkanie”),
    \item predykaty symboliczne i warunki złożone (np. „oferta dobra dla rodziny”, „oferta luksusowa”).
\end{itemize}

Przykład zastosowania reguły przybliżonej:

\begin{quote}
Oferta sprzedaży jest atrakcyjna inwestycyjnie, jeśli cena za metr kwadratowy jest niższa niż 5000 zł, nieruchomość jest w dobrym stanie lub wymaga jedynie drobnych remontów, a lokalizacja jest blisko centrum.
\end{quote}

W powyższej regule uwzględniono:
\begin{itemize}
    \item nieostre pojęcie „blisko centrum” (odległość mniejsza niż 5 km),
    \item względną ocenę „drobnych remontów” poprzez pośrednią interpretację stanu nieruchomości,
    \item próg finansowy interpretowany kontekstowo w zależności od miasta i typu nieruchomości.
\end{itemize}

\subsection*{Reprezentacja nieostrych danych i pojęć lingwistycznych}

Dane przechowywane w bazie wiedzy są przetwarzane za pomocą reguł zawierających rozmyte przedziały oraz symboliczne etykiety. Przykładowo:
\begin{itemize}
    \item „małe mieszkanie” – powierzchnia $<$ 40 m²,
    \item „blisko centrum” – odległość $<$ 5 km,
    \item „nowe mieszkanie” – rok budowy $>$ 2010,
    \item „dobra lokalizacja” – pośrednia funkcja odległości do centrum, dostępności usług i typu zabudowy.
\end{itemize}

Użycie predykatów symbolicznych pozwala systemowi na formułowanie ocen zbliżonych do sposobu myślenia użytkowników, którzy nie posługują się wartościami liczbowymi, lecz pojęciami języka naturalnego.
