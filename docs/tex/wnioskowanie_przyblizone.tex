\section{Wnioskowanie przybliżone}

Wnioskowanie przybliżone to technika stosowana w systemach ekspertowych, która umożliwia podejmowanie decyzji w warunkach \textbf{niepełnej, niepewnej lub probabilistycznej informacji}. W odróżnieniu od klasycznego wnioskowania dedukcyjnego, które bazuje na jednoznacznych faktach i ścisłych regułach logicznych, wnioskowanie przybliżone pozwala na ocenę prawdopodobieństwa wystąpienia danego zdarzenia na podstawie niepełnych danych.

W systemie ekspertowym wspierającym decyzje na rynku nieruchomości (kupno, sprzedaż, wynajem), mechanizmy wnioskowania przybliżonego są szczególnie przydatne w sytuacjach, gdy:
\begin{itemize}
    \item użytkownik dysponuje \textbf{częściowymi danymi} (np. brak pełnej historii cen w danej lokalizacji),
    \item dostępne informacje są \textbf{niepewne lub sprzeczne} (np. różne opinie ekspertów co do atrakcyjności lokalizacji),
    \item istnieje potrzeba uwzględnienia \textbf{prawdopodobieństw i statystyk} w ocenie ryzyka inwestycyjnego.
\end{itemize}

\subsection*{Przykłady zastosowania wnioskowania przybliżonego w regułach}

W systemie zastosowano reguły oparte na \textbf{ocenie prawdopodobieństwa} oraz metodach statystycznych, w tym:
\begin{itemize}
    \item reguły probabilistyczne (np. „jeśli cena nieruchomości jest niższa o 20\% od średniej rynkowej, to istnieje 80\% prawdopodobieństwa, że oferta jest atrakcyjna”),
    \item analizy statystyczne na podstawie danych historycznych (np. „prawdopodobieństwo wzrostu cen w danym regionie wynosi 65\%”),
    \item estymacja ryzyka związanego z zakupem nieruchomości w oparciu o dane z rynku (np. „jeśli lokalizacja jest odległa od centrum o więcej niż 10 km, to prawdopodobieństwo utraty wartości wynosi 30\%”).
\end{itemize}

\subsubsection*{Przykład zastosowania reguły przybliżonej:}
\begin{quote}
Oferta sprzedaży jest uznawana za ryzykowną, jeśli prawdopodobieństwo spadku cen w danej lokalizacji przekracza 50\%, a stan nieruchomości jest oceniany jako niepewny na podstawie danych z inspekcji technicznej.
\end{quote}

W powyższej regule uwzględniono:
\begin{itemize}
    \item \textbf{Prawdopodobieństwo spadku cen} obliczone na podstawie analiz rynkowych,
    \item \textbf{Niepewność stanu technicznego} wynikającą z subiektywnych ocen inspektorów,
    \item \textbf{Ryzyko inwestycyjne} jako kombinację obu czynników.
\end{itemize}

\subsection*{Reprezentacja niepewności i danych probabilistycznych}

Dane w bazie wiedzy są przetwarzane za pomocą modeli probabilistycznych oraz metod estymacji prawdopodobieństwa. Przykładowo:
\begin{itemize}
    \item „małe mieszkanie” - powierzchnia poniżej 40 m² z prawdopodobieństwem 70\% (na podstawie danych z rynku),
    \item „blisko centrum” - odległość poniżej 5 km, przy czym prawdopodobieństwo spełnienia tej cechy w danym mieście wynosi 60\%,
    \item „nowe mieszkanie” - rok budowy powyżej 2010, przy czym prawdopodobieństwo utrzymania stanu technicznego w dobrym stanie wynosi 85\%,
    \item „dobra lokalizacja” - wynik analizy danych o dostępności usług i komunikacji, estymowany z dokładnością 75\%.
\end{itemize}

Wykorzystanie metod probabilistycznych w ocenie ofert nieruchomości pozwala na modelowanie ryzyka i podejmowanie decyzji w warunkach niepewności, co odpowiada rzeczywistym wyzwaniom rynku nieruchomości.

\end{document}
