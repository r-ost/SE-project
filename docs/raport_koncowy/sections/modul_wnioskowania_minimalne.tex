\section{Moduł wnioskowania w oparciu o reguły minimalne}

Moduł wnioskowania został zaimplementowany wykorzystując teorię zbiorów przybliżonych (rough set theory) opracowaną przez Zdzisława Pawlaka. Teoria ta stanowi matematyczne narzędzie do analizy i klasyfikacji danych nieprecyzyjnych lub niepełnych.

\subsection{Wyznaczanie reduktów}

Redukt stanowi kluczowe pojęcie w teorii zbiorów przybliżonych. Jest to minimalny zbiór atrybutów, który zachowuje wszystkie relacje nierozróżnialności obecne w oryginalnym zbiorze atrybutów.

\subsubsection*{Algorytm wyznaczania reduktów}

W projekcie zastosowano algorytm wyznaczania reduktów oparty na relacji nierozróżnialności (indiscernibility relation). Algorytm został zaimplementowany w języku Python i jego pseudokod jest przedstawiony poniżej.

\begin{verbatim}
ALGORITM: Wyznaczanie_Reduktów(dane, atrybuty_warunkowe, atrybut_decyzyjny)

WEJŚCIE: 
    dane - tabela decyzyjna z obiektami
    atrybuty_warunkowe - zbiór atrybutów warunkowych
    atrybut_decyzyjny - atrybut klasyfikujący

WYJŚCIE: zbiór wszystkich reduktów i rdzeń

ALGORYTM:
    redukty := pusta_lista
    rdzeń := []
    
    // Generowanie wszystkich kombinacji atrybutów
    znaleziono_redukt := FAŁSZ
    DLA każdej długości k OD 1 DO rozmiar(atrybuty_warunkowe):
        kombinacje := generuj_kombinacje(atrybuty_warunkowe, k)
        
        DLA każdej kombinacji, zaczynając od najmniejszych:
            JEŻELI jest_reduktem(kombinacja, dane, atrybut_decyzyjny):
                dodaj kombinację do reduktów
                znaleziono_redukt := PRAWDA
        
        JEŻELI znaleziono_redukt:
            // chcemy mieć redukty o minimalnej liczbie atrybutów
            Zakończ pętlę
    
    // Wyznaczanie rdzenia jako części wspólnej wszystkich reduktów
    rdzeń := przecięcie_wszystkich(redukty)
    
    ZWRÓĆ (redukty, rdzeń)

FUNKCJA jest_reduktem(zbiór_atrybutów, dane, atrybut_decyzyjny):
    // Sprawdzenie spójności - czy nie ma sprzeczności
    JEŻELI NIE jest_spójny(zbiór_atrybutów, dane, atrybut_decyzyjny):
        ZWRÓĆ FAŁSZ
    
    // Sprawdzenie minimalności
    DLA każdego atrybutu a W zbiór_atrybutów:
        zredukowany_zbiór := zbiór_atrybutów - {a}
        
        JEŻELI zredukowany_zbiór NIE jest pusty AND 
               jest_spójny(zredukowany_zbiór, dane, atrybut_decyzyjny):
            ZWRÓĆ FAŁSZ  // nie jest minimalny
    
    ZWRÓĆ PRAWDA


FUNKCJA jest_spójny(atrybuty, dane, atrybut_decyzyjny):
    // Wyznacz relację nierozróżnialności dla danych atrybutów
    klasy_równoważności := wyznacz_klasy_równoważności(atrybuty, dane)
    
    // Sprawdź czy wszystkie obiekty w każdej klasie równoważności
    // mają tę samą wartość atrybutu decyzyjnego
    DLA każdej klasy K W klasy_równoważności:
        wartości_decyzyjne := {dane[obj][atrybut_decyzyjny] 
                                  DLA każdego obiektu obj W K}
        
        JEŻELI rozmiar(wartości_decyzyjne) > 1:
            ZWRÓĆ FAŁSZ  // sprzeczność - różne decyzje w tej samej klasie
    
    ZWRÓĆ PRAWDA


FUNKCJA wyznacz_klasy_równoważności(atrybuty, dane):
    klasy := {}
    przetworzone := {}
    
    DLA każdego obiektu obj W dane:
        JEŻELI obj NIE jest W przetworzone:
            klasa_równoważności := {obj}
            wartości_obj := pobierz_wartości(dane[obj], atrybuty)
            
            DLA każdego innego_obj W dane:
                JEŻELI inny_obj != obj AND inny_obj NIE jest W przetworzone:
                    wartości_inny := pobierz_wartości(dane[inny_obj], atrybuty)
                    
                    JEŻELI wartości_obj == wartości_inny:
                        dodaj inny_obj do klasa_równoważności
            
            // Oznacz wszystkie obiekty z tej klasy jako przetworzone
            DLA każdego o W klasa_równoważności:
                przetworzone[o] := PRAWDA
            
            dodaj klasa_równoważności do klasy
    
    ZWRÓĆ klasy
\end{verbatim}

\subsection{Wyznaczanie reguł minimalnych}

Na podstawie wyznaczonych reduktów generowane są reguły minimalne, które stanowią podstawę modułu. Wyznaczanie reguł minimalnych zostało zaimplementowane w języku Python. Wyjściem algorytmu jest zbiór reguł minimalnych w języku Prolog, które mogą być wykorzystane do klasyfikacji nowych obiektów. Zaimplementowany algorytm pozwala również na wyznaczenie reguł o zadanym poziomie dokładności, co umożliwia dostosowanie modelu do niedokładnych danych.
\\ \\
Reguły minimalne powstają według następującego schematu:
\begin{enumerate}
    \item \textbf{Wybór reduktu} - wybór pojedynczego reduktu z wyznaczonych wcześniej, dla którego będą tworzone reguły
    \item \textbf{Grupowanie obiektów} - obiekty o identycznych wartościach atrybutów z reduktu są grupowane
    \item \textbf{Formułowanie reguł} - dla każdej grupy tworzona jest reguła postaci: \\
    \texttt{JEŚLI (warunek\_1 AND warunek\_2 AND ... AND warunek\_n) TO klasa}
\end{enumerate}

\noindent W algorytmie wyznaczania reguł minimalnych zastosowano podejście przeszukiwania wszystkich możliwych kombinacji atrybutów warunkowych w redukcie. Z tego powodu złozoność algorytmu jest wykładnicza w stosunku do liczby atrybutów warunkowych w redukcie. W celu poprawy wydajności algorytmu, można byłoby zastosować heurystyki np. heurystykę Johnsona.

\subsection{Przykład działania modułu}

Weźmy następujący podzbiór atrybutów obiektu "nieruchomość" opisanego w tabeli \ref{tab:property_details}:
% 'Typ', 'Lokalizacja', 'Winda', 'Parking', 'TypParking', 'Opis'
\begin{itemize}
    \item Standard (atrybut decyzyjny): \{luksusowy, wysoki, średni, niski\}
    \item Typ: \{dom, mieszkanie\}
    \item Lokalizacja: \{centrum, obrzeża, przedmieścia\}
    \item Winda: \{tak, nie\}
    \item Parking: \{tak, nie\}
    \item TypParking: \{podziemny, zewnetrzny, brak\}
    \item Opis - tekstowy opis nieruchomości, od którego nie zależy wartość atrybutu decyzyjnego
\end{itemize}
Zadaniem rozważanego modułu wnioskowania jest wyznaczenie reguł minimalnych, które pozwolą na klasyfikację standardu nieruchomości w zależności od wartości atrybutów warunkowych. W tym celu system analizuje dane treningowe i wyznacza redukt, który jest minimalnym zbiorem atrybutów warunkowych zachowującym zdolność rozróżniania obiektów. Następnie na podstawie tego reduktu generowane są reguły minimalne w języku Prolog.

\subsubsection*{Wyznaczony redukt:} 
\begin{verbatim}
{Typ, Lokalizacja}
\end{verbatim}

\subsubsection*{Reguły minimalne zwrócone przez algorytm, zapisane w języku Prolog}
\begin{verbatim}
% autogenerated using generate_minimal_rules.ipynb
standard(Nieruchomosc, luksusowy) :-
    nieruchomosc_typ(Nieruchomosc, Typ),
    Typ = dom.

standard(Nieruchomosc, wysoki) :-
    nieruchomosc_lokalizacja(Nieruchomosc, Lokalizacja),
    nieruchomosc_typ(Nieruchomosc, Typ),
    Typ = mieszkanie,
    Lokalizacja = centrum.

standard(Nieruchomosc, niski) :-
    nieruchomosc_lokalizacja(Nieruchomosc, Lokalizacja),
    Lokalizacja = przedmiescia.

standard(Nieruchomosc, sredni) :-
    nieruchomosc_lokalizacja(Nieruchomosc, Lokalizacja),
    Lokalizacja = obrzeza.
\end{verbatim}
